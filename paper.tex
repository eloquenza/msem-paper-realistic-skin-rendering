% This template has been tested with LLNCS DOCUMENT CLASS -- version 2.20 (10-Mar-2018)

% !TeX spellcheck = en-US
% !TeX encoding = utf8
% !TeX program = pdflatex
% !TeX TXS-program:compile = txs:///pdflatex/[--shell-escape]
% !BIB program = bibtex
% -*- coding:utf-8 mod:LaTeX -*-

% "a4paper" enables:
%  - easy print out on DIN A4 paper size
%
% One can configure a4 vs. letter in the LaTeX installation. So it is configuration dependend, what the paper size will be.
% This option  present, because the current word template offered by Springer is DIN A4.
% We accept that DIN A4 cause WTFs at persons not used to A4 in USA.

% "runningheads" enables:
%  - page number on page 2 onwards
%  - title/authors on even/odd pages
% This is good for other readers to enable proper archiving among other papers and pointing to
% content. Even if the title page states the title, when printed and stored in a folder, when
% blindly opening the folder, one could hit not the title page, but an arbitrary page. Therefore,
% it is good to have title printed on the pages, too.
%
% It is enabled by default as the springer template as of 2018/03/10 uses this as default

% German documents: pass ngerman as class option
% \documentclass[ngerman,runningheads,a4paper]{llncs}[2018/03/10]
% English documents: pass english as class option
\documentclass[english,runningheads,a4paper]{llncs}[2018/03/10]

\usepackage{improved-lncs}

% Add copyright
% Do that for the final version or if you send it to colleagues
\iffalse
  %state: intended|submitted|llncs
  %you can add "crop" if the paper should be cropped to the format Springer is publishing
  \usepackage[intended]{llncsconf}

  \conference{name of the conference}

  %in case of "llncs" (final version!)
  %example: llncs{Anonymous et al. (eds). \emph{Proceedings of the International Conference on \LaTeX-Hacks}, LNCS~42. Some Publisher, 2016.}{0042}
  \llncs{book editors and title}{0042} %% 0042 is the start page
\fi

\begin{document}

\title{Realitische Echtzeitdarstellung von Haut unter Betrachtung physikalischer Grundlagen}
%If Title is too long, use \titlerunning
\titlerunning{Realitische Echtzeitdarstellung von Haut}

%Single insitute
\author{Dennis Grabowski, B.Sc.}
%If there are too many authors, use \authorrunning
\authorrunning{D. Grabowski}

\institute{Hochschule Hannover, Ricklinger Stadtweg 120, 30459 Hannover, Germany\\
\email{dennis.grabowski@stud.hs-hannover.de}\\
\url{https://f4.hs-hannover.de/}}

%% Multiple insitutes - ALTERNATIVE to the above
% \author{%
%     Firstname Lastname\inst{1} \and
%     Firstname Lastname\inst{2}
% }
%
%If there are too many authors, use \authorrunning
%  \authorrunning{First Author et al.}
%
%  \institute{
%      Insitute 1\\
%      \email{...}\and
%      Insitute 2\\
%      \email{...}
%}

\maketitle

\begin{abstract}
  Unter dem typischen Phong-Modell ist es schwer, verschiedenste Materialien wie Marmor, Wachs, Blätter oder Haut realistisch dar\-zu\-stellen.
  Zugrunde liegt, dass das Phong-Modell nur ein empirisches Model ist und keineswegs ein physikalisch haargenaues Modell darstellt, wodurch verschiedenste physikalische Effekte schlichtweg nicht nachstellbar sind.
  Diese Ausarbeitung wird ausleuchten, welche Schwierigkeiten sich ergeben, Haut realistisch darzustellen und welche Konzepte helfen, diesen Realismus wiedergeben zu können.
  Haut als Medium bietet sich besonders an, um an diese Konzepte heranzugehen, da jedem Leser hoffentlich eine realistische Präsentation dieser bekannt ist.
  Hierfür wird zunächst aufgezeigt, wie Licht mit Haut interagiert, um darauf basierend herauszuarbeiten, welche Bildverarbeitungskonzepte nötig sind, um die Defizite des Phong-Modells aufzufangen.
  Anschliessend werden physikalisch-basierte Beleuchtungsmodelle vorgestellt, welche die Reflektion des Lichts durch die Haut sowie das Eindringen des Lichts in die Haut realistisch abbilden können.
  Dazu werden die Bidirektionale Reflexionsverteilungsfunktion von Kelemen and Szirmay-Kalos, die für die realistische Oberflächenreflexion zuständig ist, und das Subsurface Scattering, welches das Eintreten und dem darauffolgendem Austreten des Lichts in die Haut möglichst realistisch simuliert, vorgestellt.
  Nachfolgend werden Bilder verschiedenster Implementationen miteinander vergliechen, um die Wirkung dieser Konzepte besser einschätzen zu können.
\end{abstract}

\begin{keywords}
  Computer graphics, physically-based rendering, BRDFs, subsurface scattering, diffusion profiles
\end{keywords}

\section{Prologue}
\label{sec:intro}

\section{Subsurface scattering}
\label{sec:subsurface}

\section{Application of aforementioned theoretical concepts}
\label{sec:application}

\section{Evaluation and discussion}
\label{sec:evalanddiscuss}

\section{Conclusion and Outlook}
\label{sec:outlook}

\nocite{*}

\renewcommand{\bibsection}{\section*{References}} % requried for natbib to have "References" printed and as section*, not chapter*
% Use natbib compatbile splncsnat style.
% It does provide all features of splncs03, but is developed in a clean way.
% Source: http://phaseportrait.blogspot.de/2011/02/natbib-compatible-bibtex-style-bst-file.html
\bibliographystyle{splncsnat}
\begingroup
  \ifluatex
    %try to activate if bibliography looks ugly
    %\sloppy
  \else
    \microtypecontext{expansion=sloppy}
  \fi
  \small % ensure correct font size for the bibliography
  \bibliography{paper}
\endgroup

% Enfore empty line after bibliography
\ \\
%
All links were last followed on April 30, 2020.
\end{document}
